%%Template creada por José Santiago Vales Mena. GitHub: 5an7y
%configuración de la pagina{
\documentclass{article}
\usepackage[utf8]{inputenc}
\usepackage[spanish]{babel}
\usepackage{geometry, hyperref, listings, ragged2e, amsmath, enumerate, minted, graphicx, xcolor, setspace, fancyhdr, amssymb, anyfontsize}

\geometry{
 letterpaper,
 left = 20mm,
 headheight=80mm
}

%}


\newcommand{\Titulo}{Aquí va tu título}
\newcommand{\Persona}{Aquí va tu nombre}
\newcommand{\Lugar}{Harvard}
\newif\ifimage
\imagefalse
\newcommand{\Imagen}{...} %Aquí va la dirección de la imagen ej. images/mariposa.png
% Si quieres una imagen en la portada, quita el "%" de la siguiente linea
%\imagetrue

%Ambientes{
    
    \newcounter{problema}[subsection]
    \newenvironment{problema}[1][]
        {\refstepcounter{problema}\par\medskip\noindent
        \large{\textbf{Problema~\theproblema. #1}} \rmfamily \begin{itshape}\normalsize}
        {\end{itshape} \medskip}

    \newcounter{respuesta}[section]
    \newenvironment{respuesta}[1][]
        {\refstepcounter{respuesta}\par\medskip\noindent
        \large \textbf{Respuesta~} \rmfamily \normalsize}
        {\medskip}

    \newcounter{subproblema}[problema]
    \newenvironment{subproblema}[1][]
        {\refstepcounter{subproblema}\par\medskip\noindent
        \thesubproblema #1 \normalsize}
        {\medskip}

    \renewcommand\thesubproblema{(\theproblema.\roman{subproblema})}

    \newcounter{subrespuesta}[respuesta]
    \newenvironment{subrespuesta}[1][]
        {\refstepcounter{subrespuesta}\par\medskip\noindent
        \thesubrespuesta #1 \normalsize}
        {\medskip}
    
    \renewcommand\thesubrespuesta{(\therespuesta.\roman{subrespuesta})}

    \newcounter{teorema}[section]
    \newenvironment{teorema}[1][]
        {\refstepcounter{teorema}\par\medskip\noindent
        \Large \textbf{Teorema~ \theteorema. #1} \rmfamily \large}
        {\medskip \normalsize}

    \newcounter{propiedades}[section]
    \newenvironment{propiedades}[1][]
        {\refstepcounter{propiedades}\par\medskip\noindent
        \large{\textbf{P~\thepropiedades. #1}} \rmfamily}
        {\medskip}

    \newenvironment{proposicion}[1][]
        {\par\medskip\noindent
        \large{\textbf{Proposición~}} \rmfamily \begin{itshape}\normalsize}
        {\end{itshape}}

    \newenvironment{demostracion}[1][]
        {\par\medskip\noindent
        \normalsize \textbf{Demostración~} \rmfamily \small}
        {\begin{flushright}$\blacksquare$ \end{flushright}\normalsize{}}
    
    \newenvironment{definicion}[1][]
        {\par\medskip\noindent
        \large \textbf{Definición~} \rmfamily \normalsize}
        {\medskip\normalsize{}}

    \newenvironment{corolario}[1][]
        {\par\medskip\noindent
        \large \textbf{Corolario~} \rmfamily \normalsize}
        {\medskip\normalsize{}}

%}

%Comandos para cosas usadas frecuentemente{
    \newcommand{\R}{\mathbb{R}}
    \newcommand{\Pn}{\mathbb{P}}
    \newcommand{\N}{\mathbb{N}}
    \newcommand{\Z}{\mathbb{Z}}
%}

\begin{document}
%Portada{
    \pagenumbering{gobble}
    \onehalfspacing
    \centering
    \fontsize{40}{48}\selectfont{ \textbf{{\Titulo}}}
    \vspace{1cm}
    
    \huge{\Persona}
    
    \ifimage
        \includegraphics[width=\linewidth]{\Imagen}
    \else
    \fi
    
    \vfill
    
    \LARGE{\today}
    
    \vspace{15mm}
    
    \Large{\Lugar}
    \newpage
%}

%Configuración del contenido {
    \normalsize
    \pagestyle{fancy}
    \rhead{\Titulo}
    \lhead{\Persona}
    \pagenumbering{arabic}
    \justify
%}

%% Aquí empiezas a escribir (en el template se escriben en otros archivos para un mayor orden, pero como gustes).


\begin{center} \section{Límites} \end{center}

\subsection{Definición}

Ya hemos visto intuitivamente lo que es un límite; sin embargo ahora nos toca definirlo formalmente.
%%Se tiene un ambiente para una definición
\begin{definicion}
    La función $f$ tiende hacía el limite $l$ en $a$, si para toda $\epsilon > 0$ existe algún $\delta > 0$ tal que, para toda $x \in \R$ que satisfaga $0 < |x - a| < \delta$ implica que $|f(x) - l| < \epsilon$
\end{definicion}

Gracias a la definición, podemos enunciar un teorema que nos sera útil 
%%Así como esta teorema igual tenemos corolario, lemma, proposición y propiedades.
\begin{teorema}
    Si $f$ tiende a $l$ en $a$ y $f$ tiende a $m$ en $a$. Entonces, $l = m$.
\end{teorema}

\begin{demostracion}
    No se demostrara en esta libreta, pero se puede revisar la demostración en el Spivak
\end{demostracion}

Y seguimos con más texto...
\begin{center} \section{Problemas} \end{center}

\subsection{Limites}

\begin{problema}
    Demuestre que $$\lim_{x \rightarrow 0^+}f\left(\frac{1}{x}\right) = \lim_{x \rightarrow \infty}f(x)$$
\end{problema}

\begin{respuesta}
    Aquí va tu solución 
\end{respuesta}

\begin{problema}
    Calcula los límites siguientes, tomando en cuenta que $\alpha = \lim_{x \rightarrow 0} \frac{\sin x}{x}$
    
    \begin{subproblema}
        $\lim_{x \rightarrow \infty} \frac{\sin x}{x}$
    \end{subproblema}
    
    \begin{subproblema}
        $\lim_{x \rightarrow \infty} x \sin(\frac{1}{x})$
    \end{subproblema}
    
\end{problema}

\begin{respuesta}
    
    \begin{subrespuesta}
        Aquí escribes la solución al primer inciso
    \end{subrespuesta}
    
    \begin{subrespuesta}
        Aquí escribes la solución al segundo inciso
    \end{subrespuesta}
     
\end{respuesta}

 
\end{document}