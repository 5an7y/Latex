
\begin{center} \section{Límites} \end{center}

\subsection{Definición}

Ya hemos visto intuitivamente lo que es un límite; sin embargo ahora nos toca definirlo formalmente.
%%Se tiene un ambiente para una definición
\begin{definicion}
    La función $f$ tiende hacía el limite $l$ en $a$, si para toda $\epsilon > 0$ existe algún $\delta > 0$ tal que, para toda $x \in \R$ que satisfaga $0 < |x - a| < \delta$ implica que $|f(x) - l| < \epsilon$
\end{definicion}

Gracias a la definición, podemos enunciar un teorema que nos sera útil 
%%Así como esta teorema igual tenemos corolario, lemma, proposición y propiedades.
\begin{teorema}
    Si $f$ tiende a $l$ en $a$ y $f$ tiende a $m$ en $a$. Entonces, $l = m$.
\end{teorema}

\begin{demostracion}
    No se demostrara en esta libreta, pero se puede revisar la demostración en el Spivak
\end{demostracion}

Y seguimos con más texto...