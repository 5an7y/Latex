\begin{center} \section{Limits} \end{center}

\subsection{Definition}

We have seen the intuitive idea of what a limit is. But in order to express what a limit is formally, we have the above definition

%% The definition have and environment

\begin{definition}
    The function $f$ goes to the limit $l$ in $a$, if for all $\epsilon > 0$ exist some $\delta > 0$ such that, for every $x \in \R$ that satisfies $0 < |x - a| < \delta$ then $|f(x) - l| < \epsilon$
\end{definition}

Thanks to the definition we can have the following theorem 
%%Like the theorem environment corollary, lemma, proposition y properties.
\begin{theorem}
    If $f$ goes to $l$ in $a$ and $f$ goes to $m$ in $a$. Then, $l = m$.
\end{theorem}

\begin{demonstration}
    The demonstration is left to the reader
\end{demonstration}

Here goes more text...